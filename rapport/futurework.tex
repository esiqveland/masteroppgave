% futurework.tex

We will here introduce and expand upon some ideas we have for future work on this project. Some ideas we did not find the time for. Others are not as of this writing possible yet.

\section{Downscaling}
Voldemort as of this project, does not support downscaling, ie. downsizing the cluster. There is work going on for this, but it is not a current priority. We would of course like to include this functionality into our project once it is possible.

\section{Data rebalancing}
Rebalancing here refers to the data balance in the cluster. With nodes being added and partitions shuffled around, the system might end up in states were the data distribution is uneven like in Table \ref{tbl:datarebalance}.
\begin{center}
\begin{table}[h]
	\begin{tabular}{|c|c|r|}

		\multicolumn{1}{c}{Node} & 
		\multicolumn{1}{c}{Partition } & 
		\multicolumn{1}{c}{data \%} \\
		\hline

		\multirow{2}{*}{0} & 0 & 12.5 \\ \cline{2-3}
		 & 1 & 12.5 \\
		 \hline
		\multirow{2}{*}{1} & 2 & 12.5 \\ \cline{2-3}
		 & 3 & 24.0 \\
 		 \hline

		\multirow{2}{*}{2} & 4 & 1.0 \\ \cline{2-3}
		 & 5 & 5.0 \\
		
		\hline	
		
		\multirow{2}{*}{3} & 6 & 20.0 \\ \cline{2-3}
		 & 7 & 12.5 \\

		\hline
	\end{tabular}
	\caption{A cluster where partitions exhibit uneven data distribution. This cluster \emph{potentially} needs rebalancing of data.}
	\label{tbl:datarebalance}
\end{table}
\end{center}

If certain partitions end up with a skewed amount of data, even shuffling partitions around to idle nodes might be fruitless. One might have to do a very expensive rehash, effectively moving the partition key space. If this could be done live in an efficient manner, it could greatly improve the balancing and performance of each node in such edge cases.

\section{Ondemand computing}
With the recent years advancement in cloud computing, especially those driven by Amazons EC2 service, it is now easy to rent computing capacity as you go and pay for what you use.
With the current design, having physical computers ready is an absolute requirement, ie. to insert a new node one needs actual hardware.

An idea we have for future work is a module to the Headmaster rebalancer that can automatically create, setup and assign Amazon EC2 machines on demand. This would allow for scaling the cluster up and down on demand with rented computing.

\section{Decision support system}
We think it would be interesting with further decision support for scaling. We currently don't do anything with regards available disk space or memory usage. 
Combining this with the possibilities in ondemand computing would be interesting.
This could allow for ordering and extending both disk space and memory in running nodes, which is currently possible with e.g. Amazon Web Services.

\section{Running with backup nodes}
One could assign a certain znode path or a certain property with new nodes in \texttt{/active} to add a node as a backup node.
A backup node can be a queued node that is kept ready for use in case another node dies permanently. E.g. if a node has a permanent failure, this backup node is put in to replace it immediately. A challenge here could be false positives, replacing a node before we need to.

One could also see backup nodes waiting on znodes in \texttt{/active} to disappear, then immediately replace them. This does not seem like very efficient use of resources though.

\section{Real world testing}
As we have done this project without a real application using Voldemort, we do not have access to realistic test data. We feel a few more real world tests should be done before our management system can be deployed to a live production environment.

It would also be interesting to apply it to a already running instance and see if and what improvements could be made, both to the code and to the running cluster.
Seeing how and if a running real world cluster can gain anything from the automatic load adaption would help furthering this kind of automatic management.

