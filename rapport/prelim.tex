\section{Prelim}

\subsection{Extending filesystem}
for more storage, we had to extend the filesystem:
http://archlinuxarm.org/forum/viewtopic.php?f=31&t=3119&start=20#p33507

To take advantage of the entire SD card.

\subsection{Setup of Oracle JDK7/8}
For hardfloat and different other things.

JDK7 and especially JDK8 has shown good performance on RPI even compared to native code.

https://jdk8.java.net/download.html

For early access images of JDK8, b123 at time of writing (15.1.2013).

Following the guide at: http://www.savagehomeautomation.com/raspi-jdk8

mkdir -p -v /opt/java

tar xvzf ~/jdk-8-ea-b106-linux-arm-vfp-hflt-04_sep_2013.tar.gz -C /opt/java

To complete the JDK installation we need to let the system know there is a new JVM installed and where it is located.  Use the following command to perform this task.


In the line started with "SRV_JAVA_OPTS" find "-Xmx512m" and change it to a lower value than the physical ram available on your board, 412MB worked well for me.


\subsection{Configuring Voldemort}

generate\_cluster.py
Python script that aids in the generation of Voldemort cluster.xml config file.

This file lists all the nodes in the cluster and the 
\subsection{Running Voldemort} 

As expected, lots of problems arose when trying to deploy and run voldemort on the tiny embedded architecture.

JNA.jar included did not have linux-arm precompiled binaries for the native method calls used in the software.

The call done by JNA is to mlockall:

Since 0.6.2: JNA for mlockall. This prevents Linux from swapping out parts of the JVM that aren't accessed frequently. Chris Goffinet reported a 13% performance improvement in his tests from this change. CASSANDRA-1214 
(https://journal.paul.querna.org/articles/2010/11/11/enabling-jna-in-cassandra/)
mlockall locks the memory that JVM uses as a heap into memory, so that it is not swapped out.

