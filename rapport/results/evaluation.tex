% evaluation.tex

We will first mention a few of the general experiences we got doing this project.
We will then discuss the response time results. In Section \ref{eval:throughput} we will discuss the single node throughputs.
Section \ref{eval:balance} discusses the results of the rebalance operations and measured performance scaling.

\section{Experiences}
\subsubsection{Network limits}
We did not have access to a lot of different hardware while doing this project. Our computers proved quite fast, with each individual node pushing 50-70 MB per second. This made us quickly reach the network speed limit in our benchmarking system, receiving over 130MB/s second from the cluster while benchmarking. This limit was reached around 60k requests/s, so this is our practical, and theoretical, maximum throughput of 1024 kB values. On a 100mbit connection we could only send or receive about 5k requests/s, or roughly 12 MB/s. 

\subsubsection{Server side caching}
While testing, we noticed no real difference between a 12MB and 2GB database cache on our Voldemort servers. This might suggest there is a problem with the cache setting in the software. All of our hardware run on SSDs, but we still consider this result surprising. There could be more work done here on largely different sizes of data sets, but this was not our main focus.

\subsubsection{Other notes}
Overall we found the performance to be very consistent. Our results were on large very reproducible and saw little variance between benchmark runs.

\section{Response time}
During these tests we see a lot of different behavior for the different hardware. This would suggest there can be a lot to gain on better balancing between heterogeneous nodes.


The i7 with 4 cores clearly does the best, displaying same performance for both code bases and sub 5 ms response time for 99.9\% of requests.

\section{Throughput}
\label{eval:throughput}

\section{Scaling and balancing}
\label{eval:balance}
As we can see, to achieve full potential scaling some skewing of partitions is required.

Figure shows that this can also be achieved by automatic methods.

