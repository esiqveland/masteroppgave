\section{Summary}
We started out trying to get a cluster of Raspberry PIs to run Voldemort. After a few weeks of research, debugging and hair pulling, we concluded that the PIs were simply too weak and we looked elsewhere for faster hardware. 

Through our project we have fundamentally changed how Voldemort stores configuration data and utilized ZooKeeper to provide powerful new features. We have created Headmaster to administrate the cluster and allow for new nodes to automatically be included in the cluster. Finally we have created a monitor service using the SigarAPI and the decision taker StatusAnalyzer. Through our experiments we have verified that our implementation does not suffer any noticeable performance loss and successfully migrates partitions to achieve a more balanced cluster. 

\section{Conclusion}

Management of a cluster consisting of lots of nodes is cumbersome and surprisingly error prone. We have shown that a more automatic approach can be viable, especially in terms of delivered performance. Automating management and scaling of such systems can be immensely helpful, but there is still work to be done before it can manage a live production system.

Management of distributed systems becomes increasingly difficult as the number of nodes increases. Scaling a system to meet computing requirements is also not a trivial task as demands change over time. We have successfully created a management system for Voldemort that automates some of these tasks, automatically adjusting the system according to its current status and assists in maximizing throughput. 


