\section{Summary}
We started out trying to get a cluster of Raspberry PIs to run Voldemort. After a few weeks of research, debugging hair pulling we concluded that the PIs were simply too weak and we looked had to look elsewhere for faster hardware. Through our project we have fundamentally changed how Voldemort stores configuration data and utilized ZooKeeper to provide powerful new features. We have created Headmaster to administrate the cluster and allow for new nodes to automatically be included in the cluster. Finally we have created a monitor service using the SigarAPI and the decision taker StatusAnalyzer. Through our experiments we have verified that our implementation does not suffer any noticeable performances loss and successfully migrates partitions to achieve a more balanced cluster. 

\section{Conclusion}

Disposisjon:
	
	- Management is cumbersome. Requires lots of manhours
	- Many system have their own implementations of automatic scaling (Dynamo, Cassandra)

	- We have successfully created such a system for Voldemort. 

	- Automatic management might not have a permanent place in production systems, but they could be used to get a greater understanding for cluster balance and monitor patterns.

	Frequently moving large data sets is very ineffective. 

	- 

Management of distributed systems becomes increasingly difficult as the number of nodes increases. Scaling a system to meet computing requirements is also not a trivial task as demands change over time. We have successfully created a management system that automates some of these tasks. We are happy with what we have implemented and although. 





